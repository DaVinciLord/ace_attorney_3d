\section{Autres}
	
	Dans cette partie, nous allons présenter les problèmes que nous avons rencontrés au cours de ce projet, les améliorations possibles pour pouvoir le continuer et enfin les compétences que nous avons acquises lors de la réalisation des différentes parties du projet.
	
\subsection{Problèmes rencontrés}
	Dans ce projet nous avons rencontré différentes difficultés. Tout d'abord, il a fallu s'initier à l'utilisation d'OpenGL. Certains aspects furent compliqués à réaliser, notamenent réussir une projection au bon endroit avec volume de vue adéquat. Mais aussi créer des solides avec des faces dans le bon sens, des textures correctement appliquées. Il nous a fallu aussi gérer l'éclairement, la décompositon de certains éléments en pavage pour un rendu plus lisse. La lumière diffuse a aussi été dûre à gérer, longtemps les faces opposées à la lumière sont restées noires. Il a aussi fallu gérer un grand nombre de textures, leur application dans le bon sens sur chacunes des figures. De plus, il a fallu comprendre l'utilisation des quadriques et l'appliquation des textures sur eux. Le witness stand a requis l'usage de formules de trigonométrie afin d'obtenir une disposition homogène des piliers.
	Du coté de la lecture du script, les complications ont été d'encapsuler les accès mémoire dans un thread, pour ne pas ralentir la fenêtre graphique.
	 Enfin, la modularisation du programme en fichiers disjoints a été longue à faire à cause des étroits liens entre les globales et leurs initialisations, ainsi que leur usage dans des fonctions.
\subsection{Idées d'améliorations}
	Notre projet est loin d'être exhaustif, et il y a beaucoup d'amélioration qui peuvent être faites. En premier lieu, certaines des musiques du jeu n'ont pas été implantées, pour ne pas trop allourdir le projet, ce qui laisse quelques blancs dans le jeu. Ensuite, les textures et même les formes des personnages ne sont pas superbes, et mériteraient d'être refaites et améliorées. Un plafond, manquant pour l'instant, sera ajouté. La possibilité de sauvegarder sa progression, la fait que les réponses érronées soient pénalisées et mênent à l'echec du jeu si on en fait trop, ainsi que l'affichage du dossier des preuves, sont des fonctionalités présentes dans le jeu d'origine et qui seront aussi implémentées. Enfin, le netoyage du code, et des évenements personalisés rendrons le jeu plus propre.


\subsection{Compétences acquises}
	Les compétences acquisent durant la réalisation de ce projet sont la manipulation des bibliotheques OpenGL, et d'une grande partie de leurs fonctionalités.
	L'usage de la SDL, version 2, notamment de la file d'êvenement qui a permis de gerer de manière efficace le deroulement du jeu.
	Le renforcement de notre connaissance des thread POSIX.
	La manipulation des chaines de caractères en C afin de pouvoir lire un script, depuis plusieurs fichiers sources, et effectuer des actions en conséquant.