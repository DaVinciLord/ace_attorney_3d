\section{Autres}
	
	Dans cette partie, nous allons présenter les problèmes que nous avons rencontrés au cours de ce projet, les améliorations possibles pour pouvoir le continuer et enfin les compétences que nous avons acquise lors de la réalisation des différentes parties du projet.
	
\subsection{Problèmes rencontrés}
	Dans ce projet nous avons rencontré différentes difficultés. Tout d'abord, il a fallut s'initier à l'utilisation d'OpenGL. Certains aspects furent compliquer à réaliser, notamenent réussir une projection au bon endroit avec volume de vue adéquat. Mais aussi créer de solides avec des faces dans le bon sens, des textures corectement appliquées. Il nous a fallut aussi gerer l'éclairement, la décompositon de certains éléments en pavage pour un rendu plus lisse. La lumière diffuse a aussi été dûre à gérer, longtemps les faces opposées à la lumière sont restées noires. Il a aussi fallut gérer un grand nombre de textures, leur application dans le bon sens sur chaqu'unes des figures. De plus, il a fallut comprendre l'utilisation des quadriques et l'appliquation des textures sur eux. Le witness stand a requis l'usage formules de trigonometrie afin d'obtenir une disposition homogène des piliers.
	Du coté de la lecture du script, les complications ont été d'encapsuler les accès mémoire dans un thread, pour ne pas ralentir la fenetre graphique.
	 Enfin, il modularisation du programme en fichiers disjoints a été longue à faire à cause des étroits liens entre les globales et leurs initialisation, et leur usage dans des fonctions.
\subsection{Idées d'améliorations}
	Notre projet est loin d'être exhaustif, et il y a beaucoup d'amélioration qui peuvent être faites. En premier lieu, certaine des musique du jeu n'ont pas été implentées, pour ne pas trop allourdir le projet, ce qui laisse quelques blanc dans le jeu. Ensuite, les textures et même les formes des personage ne sont pas superbes, et mériteraient d'être refaites et améliorées. Un plafond, manquant pour l'instant, sera ajouté. La possibilité de sauvegarder sa progression, et la fait que les réponses érronées soit pénalisée et mênent à l'echec du jeu, si on en fait trop, et l affichage du dossier des preuves, sont des fonctionalités présente dans le jeu d'origine et qui seront aussi implémentées. Enfin, le netoyage du code, et des évenements personalisés rendrons le jeu plus propre.


\subsection{Compétences acquises}
	Les compétences acquise par la réalisation de ce projet sont la manipulation des bibliotheques OpenGL, et d'une grande partie de leurs fonctionalités.
	L'usage de la SDL, version 2, notemment de la file d'êvenement qui a permise de gerer de manière efficace le deroulement du jeu.
	Le renforcement de notre conaissance des thread POSIX.
	La manipulation des chaines de caractères en C afin de pouvoir lire un script, depuis plusieur fichier source, et effectuer des actions en conséquant.