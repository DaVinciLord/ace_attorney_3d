\section{Les fonctionnalités du script}

	Le but de notre projet était de revisiter l'univers d'Ace Attorney et ce de deux manières différentes:
d'une part  grâce à la réalisation graphique, que nous verrons plus tard dans ce rapport, et d'autre part par le suivi du script.\newline

	En effet, comme dit en introduction, le script a une place majeure dans le déroulement du jeu car c'est lui qui fait avancer ou non l'intrigue.\newline
Pour des soucis de temps, nous avons choisi de ne ré-implémenter que la première affaire du premier jeu.\newline

	Cette partie du rapport se découpe de la manière suivante: dans un premier temps, nous verrons comment les différents fichiers de script s'organisent entre eux.Dans un second temps, nous décrirons les structures utilisées pour la bonne gestion des fichiers de textes selon la situation. Enfin, nous verrons comment l'implémentation de cette partie en elle même.\newline

\subsection{Les fichiers}
Tout d'abord, les fichiers de textes sont tous placés dans le dossier ./text\_files/ afin de bien séparer les morceaux de l'application finale.
Ces fichiers s'articulent donc en plusieurs catégories qui sont les suivantes:

\begin{itemize}
	\item evidences.txt -> ce fichier sert à lister toutes les preuves ainsi qu'une petite description de celles-ci
	\item script.txt -> le script général. C'est le fichier parent à tous les autres.
	\item qcms/ -> les différents qcms du script en cours
	\item cross\_exams/ -> les fichiers représentant la contre-argumentation en cours
	\item testimonies/ -> les différents témoignages du script en cours.
\end{itemize}

	Dans le fichier script.txt, et ceux qui en découlent, nous trouveront un certains nombre de balises, délimitées par "[" et "]", qui indiquent une certaine action à effectuer en fonction du contenu de la balise (que ce soit le lancement d'une musique ou d'une autre partie du script), sauf quand un personnage parle: cette information est indiquée par une "*" avant le nom du personnage, et les lignes de dialogues qui suivent sont affichées sur la console.


\subsection{Les structures}

	Dans cette partie, nous allons décrire les différentes structures associées à la gestion du script et à sa bonne exécution.

\begin{verbatim}

struct key_value {
	char *key;
	char *value;
};
\end{verbatim}

	Cette structure est la structure classique pour contenir un couple de clé-valeur.

\begin{verbatim}

struct qcm_struct {
	char *talking;
	char *question;
	char *answer[5];
	int nb_proposition;
	struct key_value proposition_case[5];
	char *case\_files[5];
};
\end{verbatim}

	La structure d'un QCM renseigne toutes les informations utiles à la progression de celui-ci, c'est à dire le personnage posant la question, la question en elle même, les réponses possibles, le nombre de propositions, un couple de clé/valeur associant comme clé "CASE\_i" à la valeur de texte qui sera affiché dans le menu du QCM. Enfin, la structure enregistre les différents fichiers qui seront lus quand une réponse est proposée.

\begin{verbatim}


struct evidence {
	char *evidence_name[30];
	char *evidence_type[30];
	char *evidence_other[30];
	char *evidence_description[30];
	int evidence_active[30];
	int nb_evidence;
};
\end{verbatim}

	Cette structure permet l'enregistrement des informations sur les preuves contenues dans le fichier "evidences.txt", avec notamment son nom, son type, sa description, si elle est active ou non et ses informations complémentaires. La structure renseigne également sur le nombre de preuves totales du fichier.

\begin{verbatim}

struct cross_exam_struct {
	char *answer_line[15];
	int evidence_to_show;
	char *lines[20];
	char *extras[20];
	char *name;
	int current_line;
	int nb_lines;
	int nb_extras;
};

\end{verbatim}

	Dernière structure de données permettant l'emcapsulation des informations pour le bon déroulement d'une contre-argumentation. Elle possède toutes les lignes du témoignage, les lignes correctes sur lesquelles on peut présenter une preuve et si le numéro de la preuve (evidence\_to\_show) est la bonne fini la contre-argumentation. Après avoir passé toutes les lignes du témoignage, on affiche les lignes "extras", c'est à dire, un petit dialogue qui se passe après mais qui en conclue pas cette phase de jeu.
	
\subsection{L'implémentation de la lecture du script}
	
	En premier lieu, pour éviter que la lecture du script n'empiète sur l'affichage de la scène, et donc pouvoir utilisé les différentes touches claviers pour effectuer les actions que nous avons définies, nous avons dû l'encapsuler dans un thread POSIX afin de vraiment dissocier le thread principal gérant l'affichage de la scène et le thread utilisateur permettant la lecture des différents fichiers.\newline
	
	Dans un second temps, dès qu'une ligne indiquant le changement de fichier (par les balises [QCM] ou [CROSS\_EXAM]), un nouveau fichier est lu, mettant en suspend la lecture en cours. Des fonctions sont prévues pour ces deux cas là, car ils sont différents d'une simple lecture puisqu'ils répondent à un certain gameplay.\newline
	
	Dans l'implémentation proposée, on vérifie également la ligne du script lue pour savoir le comportement a adopté. En effet, si la ligne commence par une étoile, on indique qu'il faut changer le point de vue de la caméra pour pointer sur le personnage qui parle.\newline
	Pour des balises de type [START\_MUSIC\_TRIAL] ou [JUDGE\_HAMMER] cela lance la musique ou bien la petite animation associée à ce type d'évènement.\newline
	
	Pour parvenir à indiquer à la partie graphique qu'un changement doit être opéré, et ce parce que les deux parties de l'application sont sur deux threads différents, nous avons dû nous servir de la fonction SDL\_PushEvent(SDL\_Event *event) permettant d'empiler sur la pile des évènements de la SDL pour que la partie graphique puisse se mettre à jour en fonction des données du script. Nous avons pu utiliser cette fonction puisque la documentation de la SDL indique que celle-ci est "thread-safe" et que les changements opérés sur un thread seront visibles sur un autre.